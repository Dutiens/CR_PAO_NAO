\section{Présentation du projet à des élèves extérieurs}
\label{sec:Présentation du projet à des élèves extérieurs}
  \par Au cours des deux derniers mois du \pao\ nous avons eu l'occasion de présenter à deux reprises notre projet devant des élèves extérieurs.\\

  \subsection{Présentation du 22 avril}
  \label{sub:Présentation du 22 avril}
    \par La première présentation a eu lieu face à des élèves de seconde visitant l'établissement.
    Pour l'occasion, un diaporama expliquant rapidement et de la manière la plus simple le but de notre travail a été conçu.\\
    Nous avons fait le choix d'utiliser le framework JavaScript Reveal.js, pour ce diaporama. 
    Ce framework permet de faire des presentations interactives de manière relativement aisée en HTML / CSS, puis de les visualiser depuis son navigateur. Cependant comme nous ne l’avions pas utilisé depuis un moment, cela a nécessité un certain temps de réadaptation.
    L'avantage de cette solution est qu'elle nous permettait d'utiliser un dépôt git, facilitant la mise en commun et les mises à jour de notre travail.\\
    Trouver comment aborder le sujet du projet pour la présentation devant les lycéens et essayer d’adapter son discours en conséquence n’a pas été si aisé.
    De plus, une démonstration du Nao jouant à Pong était requise.
    Cependant, à la date de la présentation, le projet était encore loin de son résultat final, et le code que nous avions ne nous permettait pas de faire une démonstration convenable de notre travail.
    Nous avons donc été contraint de réutiliser le travail effectué par les élèves travaillant avant nous sur ce PAO.\\

  \subsection{Présentation du 10 mai}
  \label{sub:Présentation du 10 mai}
    \par La seconde présentation s'est déroulée devant des élèves en CM1/CM2. \\
    Dans le cadre d'un projet personnel (Chimie Kids) des étudiants en CFI3 nous ont contacté afin que nous leur présentions le département ASI ainsi que notre PAO.
    Cette présentation étant de courte durée et face à d'aussi jeunes élèves, aucun diaporama n'était requis. Nous avons quand même dû donner quelques explications et il se trouve qu'adapter notre discours pour les écoliers s'est révéler encore plus difficile que lors de la présentation face aux lycéens.
    Nous avons également dû effectuer une démonstration du Nao jouant à Pong.
    Cette fois, notre avancement dans le projet nous a permis d'utiliser notre code.
\pagebreak
