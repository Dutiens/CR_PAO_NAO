\section{Présentation du projet}
\label{sec:Présentation du projet}

\par La faisabilité du projet ayant été prouvée, notre but était donc d'améliorer le fonctionnement afin d'obtenir de meilleures performances. Nos objectifs étaient donc de rendre un code propre et modulaire, d'améliorer la reconnaissance et l'intelligence artificielle du Nao.

\subsection{Propreté et modularité du code}
\par Le dépôt du projet tel que nous l'avons récupéré contenait des fichiers  de méthodes de reconnaissance différentes, les traitements dispersés dans différentes parties et le tout n'était pas vraiment organisé.

\par Notre but principal lors de ce PAO était donc de restructurer le projet afin de le rendre plus modulaire et facilement maintenable en effectuant une meilleure répartition des différents éléments nécessaires au bon fonctionnement du projet.
    
\subsection{Amélioration reconnaissance}
\par La reconnaissance prenant trop longtemps pour que les actions du Nao soient performantes, un objectif important du projet était donc de réduire le temps du processus de reconnaissance tout en conservant les performances de détection.

\subsection{Amélioration comportement}
\par En l'état les performances du Nao étaient loin d'être convaincantes. Il fallait donc développer le comportement afin d'obtenir de meilleurs résultats.
\pagebreak
