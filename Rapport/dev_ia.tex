\subsection{Module IA}
\label{sub:Module IA}
	\par Nous avons décidé pour ce PAO d'améliorer l'IA du NAO qui se contentait auparavant de s'aligner selon l'axe vertical avec la balle.

	\par L'IA que nous avons développée suit des règles simples qui sont les suivantes :

	\begin{itemize}
		\item Si la balle se dirige vers la barre adverse, l'action renvoyée doit permettre au NAO de se positionner au milieu du terrain selon l'axe	vertical.
		\item Sinon à l'aide de deux positions de la balle, l'angle de déplacement de celle-ci est calculé, puis la position où cette dernière est censée arriver au final.
		Ainsi l'IA peut être capable de se positionner correctement.
	\end{itemize}

	\par Nous avons admis qu'un rebond sur un mur avait pour effet sur l'angle de	l'inverser verticalement (Exemple: Un angle de 45\degre devient 135\degre).

	\par Du fait de notre conception, où il a été décidé que le robot devait prendre un temps de montée ou de descente pour agir, l'IA renvoie l'action correspondante à l'indication de montée ou descente pendant un certain temps.
	Pour cela il lui est nécessaire de calculer le temps de l'action afin	d'atteindre la position voulue.

	\par Il s'est révélé par la suite que la vitesse passée à l'IA que nous obtenons ne	semble pas cohérente, ce qui fausse en partie les résultats pour l'action.
	De plus	le comportement donné dans le programme principal (calcul avec la précédente balle qui a fait l'objet d'un traitement avec la plus récente détectée) ne permet pas d'obtenir régulièrement des résultats cohérents à cause des rebonds et des remises	en jeu qui faussent l'estimation du déplacement de la balle.
\pagebreak
