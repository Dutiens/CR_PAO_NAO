\subsection{Module nao\_reco\_tester}
\label{sub:Module nao_reco_tester}
  \par Au cours de notre développement, deux méthodes de reconnaissance ont fait leur apparition.
  Afin de tester ces deux méthodes un ensemble d'images de tests où les positions étaient connues nous a été fourni.

  \par Afin d'effectuer nos tests, nous avons donc développé un outil (Fichier statistique\_reconnaissance.py)nous permettant de récupérer les différentes erreurs produites par notre reconnaissance dans un fichier au format CSV.
  Les données sauvegardées dans le fichier CSV sont les suivantes :

  \begin{itemize}
      \item Chemin vers l'image;
      \item Erreur pour la balle sur l'axe horizontal;
      \item Erreur pour la balle sur l'axe vertical;
      \item Erreur pour la barre gauche;
      \item Erreur pour la barre droite;
      \item Erreur globale moyenne;
      \item Temps mis pour la reconnaissance du terrain;
      \item Temps mis pour la reconnaissance des éléments
  \end{itemize}

  \par Il s'est avéré que les données qui nous avaient été fournies ne correspondaient pas aux positions sur les images.
  Il nous était donc impossible d'effectuer des tests avec celles-ci.

  \par Nous avons donc décidé de développer une deuxième version de l'outil (Fichier nao\_capture\_pong.py).
  L'objectif de cette deuxième version était de générer des états d'un jeu de Pong où les coordonnées sont connues et de faire en sorte que le NAO prenne une image du terrain et effectue sa reconnaissance.
  Ainsi nous obtenions les données nous permettant de calculer l'erreur produite par nos méthodes de reconnaissances.

  \par Malheureusement, il ne nous a pas été possible de terminer cette version de l'outil du fait du temps qui nous était imparti pour effectuer le projet.
  Nous avons donc préféré concentrer nos efforts sur des points plus importants du projet.
  Ainsi, l'outil dans sa deuxième version est actuellement capable de générer un état du jeu (à l'aide de la librairie PyQt4).
  La partie manquante étant la capture d'image par le robot ainsi que l'enregistrement des données.
  Le programme devra donc prendre à terme en tant que paramètre l'IP du NAO.

  \par L'exécution de ces outils se fait à partir du dossier 'nao' à l'aide des commandes suivantes : \\
  - \textit{python2 chemin\_vers\_statistique\_reconnaissance.py [IP du NAO]}\\
  - \textit{python2 chemin\_vers\_nao\_capture\_pong.py [IP du NAO]}


  \par Concernant le premier programme, il est a noté que les images doivent se situer dans le dossier 'nao' afin de fonctionner.
\pagebreak
