\section{Conception}
\label{sec:Conception}

\par La conception inhérente au projet à été l'un des points les plus importants pour nous au cours de ce PAO. En effet, notre but étant de fournir un code bien organisé, propre et fonctionnel, il a été important de privilégier la création de diagrammes nous permettant de nous donner une vue d'ensemble du projet.
\par Ainsi deux diagrammes ont étés produits. Tout d'abord un diagramme de classes (annexe \ref{sec:Diagramme de classes} page \pageref{sec:Diagramme de classes}) présentant la répartition du code d'un point de vue axé objet contrairement à ce qui avait été produit précédemment. Et un diagramme de séquence (annexe \ref{sec:Diagramme de séquence} page \pageref{sec:Diagramme de séquence}) présentant le déroulement que nous voulions intégrer au programme principal du projet (Main).
\par Le diagramme de classes s'est vu révisé plusieurs fois afin de s'assurer de la conformité selon la norme UML ainsi que de la logique nécessaire au bon fonctionnement du programme final.
\par Le diagramme de séquence quand à lui, comme dis précédemment, à été produit seulement afin de réfléchir sur ce que devait effectuer le programme en accord avec le diagramme de classes.

