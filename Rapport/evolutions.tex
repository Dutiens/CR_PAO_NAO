\section{Evolutions potentielles}
\label{sec:Evolutions potentielles}

  \subsection{Amélioration des performances}
  \label{sub:Améliorations des performances}
    \par Bien que nous ayons considérablement augmenté le temps de réaction du NAO, il pourrait être intéressant de gagner encore quelques precieux dixièmes de secondes.\\
    Nous avons constaté que l'un des principaux facteurs de ce temps de réaction était l'envoi par le réseau wifi, de l'image capturée par le NAO, jusqu'à l'ordinateur effectuant les calculs.
    Ce temps d'envoi a déjà été considérablement réduit en utilisant l'emtteur wifi local \"foreanao\", et non le réseaux wifi public de l'INSA.
    Il est cependant indéniable qu'à l'heure actuel le moyen permettant le transfert de fichier dans un réseau local serait l'utilisation de cables ethernet.
    Passer à un réseau local filaire pourrait donc être une première amélioration matérielle.\\
    Une seconde amélioration serait le développement d'un serveur gérant l'IA et les decisions du NAO, tandis que la reconnaissance serait effectuée directement sur le NAO.
    En effet, bien que certains calculs puissent être assez couteux s'ils sont effectués sur le NAO,
    ce temps serait largement compensé par l'envoi des informations utiles extraites sur les images capturées au lieu des images complètes, comme c'est actuellement le cas.
    Un ordinateur restant toujours plus puissant que le NAO, il est plus interressant d'effectuer le reste des traitements sur le serveur, surtout dans le cas du développement d'une IA plus complexe.\\

    \par Le principal problème de l'IA actuelle est le fait qu'elle ne base ses decisions que sur les deux dernières positions de balle qu'elle obtient. Cela implique
    \begin{figure}[!h]
      \caption{illustration du problème IA}
      \centering
      \includegraphics[width=\textwidth]{Images/illustrationProblemeIA.png}
    \end{figure}
          \begin{itemize}
            \item pouvoir détécter les changements de direction
            \item à chaque changement de direction de la balle (J1 -> J2 ou J1 <- J2), sauvegarder l'ensemble des positions de la balle
            \item déterminer la trajectoire complète de la balle
            \item à chaque nouvelle position de balle, vérifier si la balle est bien sur la trajectoire (seuil compris)
            \item si ce n'est pas le cas revérifier la trajectoire calculée à partir de l'ensemble de balles stockées, et supprimer les valeurs incohérentes
            \item effacer la liste de balle à chaque changement de direction / remise en jeu
            \item essayer detecter si le coup va être gagnant afin de prévoir les remises en jeu
          \end{itemize}


  \subsection{Amélioration du comportement}
  \label{sub:Amélioration du comportement}
  \par Revoir la position du NAO
        \begin{itemize}
          \item ne pas chercher à se repositionner si la position détéctée au lancement du programme est correcte
          \item chercher l'écran du regard à l'initialisation
          \item desasservir les moteurs du bras droit et des jambes afin d'éviter les surchauffes
          \item effectuer des \"dandinements\" sur un maximum de parties du corps possible, encore une fois dans le but d'éviter les surchauffes
        \end{itemize}

  \subsection{Autres améliorations}
  \label{sub:Autres améliorations}
    \par Revoir la conception du socle bien trop abimé et instable
    \par Détécter quelle barre est controllée par le NAO et jouer en conséquence
    \par Développement d'un module de test affichant en temps réel les positions estimées des éléments du jeu estimées par la reconnaisance, et les positions d'arrivées prévues par l'IA
\pagebreak
