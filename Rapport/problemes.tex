\section{Problèmes rencontrés}
\label{sec:Problèmes rencontrés}

  \par Au cours de ce projet nous nous sommes retrouvé face à un certain nombre de difficultés, dans des domaines assez variés compte tenu de l'évolution du projet.
  En effet, les problèmes que nous avons pu rencontrer au cours des premières semaines concernaient plus des problèmes matériels,
  tandis que ceux rencontrés durant les dernières semaines étaient plus de l'ordre de contraintes de développement.\\
  Les sous-parties suivantes traiterons donc, par ordre chronologique, de ces problèmes auxquels nous avons dû faire face.\\

  \subsection{Semaine du 18 janvier au 1 fevrier}
    \label{sub:Semaine du 18 janvier au 1 fevrier}
    \begin{itemize}
      \item Les dépendances du projet ne nous ayant pas été fournies, nous avons dû les récupérer directement depuis internet.
      Or l'API NaoQI, indispensable pour interagir avec le robot, necessite l'obtention d'un statut de développeur au sein de la plateforme aldebaran, statut qui nous a été assez difficile à obtenir.
      \item Nous avons mis un certain temps à nous appercevoir que l'API NaoQI repose sur une ancienne version du langage Python (2.7).
      % NOTE: Flo t'as certainement un truc à ajouter ici pour les contraintes que ça a pu apporter
      \item Le code fourni n'était pas fonctionnel: il contenait un certain nombre de bugs que nous avons du corriger.\\
    \end{itemize}



  \subsection{Semaine du 2 fevrier au 29 fevrier}
  \label{sub:Semaine du 2 fevrier au 29 fevrier}
    \begin{itemize}
      \item L'installation de Python, et surtout de toutes les librairies nécessaires au fonctionnement du projet, sous Windows et sur machine virtuelle Linux s'est révéler être assez fastidieuse.
      En effet, il semblerait que l'OS du NAO n'ait jamais été mis à jour. Cela implique l’utilisation de librairies obsolètes depuis plus d’un an et demi, souvent difficile à obtenir et parfois peu documentées.
      \item A l'exception de Florian qui pratiquait déjà le Python depuis un certain temps, la nouveauté de ce langage nous freine un peu.
      \item Nous éprouvons parfois quelques difficultées à communiquer correctement entre nous, afin de transmettre au mieux nos idées.
      Certains points de conception du projet on donc été source de débats animés.\\
    \end{itemize}



  \subsection{Semaine du 1 mars au 15 mars}
  \label{sub:Semaine du 1 mars au 15 mars}
    \begin{itemize}
      \item Une fois la mise à jour installée sur le NAO rouge, nous avons eu plusieurs problèmes qui ont mis un certain temps à être corrigés.
      En effet lors de chaque redémarrage du robot il nous été renseigné que le NAO n’arrivait pas à se connecter aux services web.
      Bien qu’il fût possible de nous connecter en SSH au NAO (qui était par conséquent connecté au réseau local),
      il nous était impossible d’accéder à l’interface web, une authentification sur les services web d’aldebaran étant requise à l’issue de la mise à jour.
      C’est finalement Florian qui a trouvé d’où venait le problème : l’heure et la date du robot s’initialisaient à sa valeur d’usine, lors de chaque redémarrage.
      Une fois le correctif appliqué, nous avons enfin pu accéder à l’interface web du NAO afin de lui appliquer une configuration de base correcte (langue, position à l'issue du démarrage).
      \item Le NAO bleu n’a jamais été mis à jour (sa version est antérieur à celle du NAO rouge avant la mise à jour),
      et la version du logiciel choregraphe (permettant d’appliquer cette mise à jour) nécessite une clé que nous ne possédons pas.
      \item Nous avons éprouvé quelques difficultés à nous souvenir de la façon de concevoir les diagrammes de classes,
      notamment sur le sens des compositions et des cardinalités, qui n'ont pas encore été abordés en cours d'UML.
      \item N'ayant encore manipulé aucun logiciels de conception de diagrammes de classes (tous ceux que nous avions pû concevoir par le passé étaient sous format papier),
      quelques recherches ont du être effectués pour trouver un logiciel gratuit répondant à nos besoins.
      Après quelques tests, notre choix c'est finalement porté sur le logiciel VisualParadigme.
    \end{itemize}



  \subsection{Semaine du 16 mars au 29 mars}
  \label{sub:Semaine du 16 mars au 29 mars}
    \begin{itemize}
      \item Les fonctionnalités du logiciel VisualParadigme étant beaucoup trop bridées dans sa version d'essai, la conception a du être réeffectuée sur le logiciel DIA.
      \item Les diagrammes de séquences n'ont pas encore été abordées en cours d'UML, et nous ne les avions encore jamais rencontrés par le passé.
      Comprendre leur fonctionnement à donc necessité un certain temps.
      \item Trouver un logiciel de conception de diagramme de séquence.
      En effet nous l’avons d’abord recopié (la conception a eu lieu sur le tableau de la salle de projet) sur le logiciel DIA,
      cependant ce dernier ne répondant pas à certaines normes d’UML2 la reproduction n’était pas parfaite.
      Nous sommes donc passés par la plateforme genmymodel utilisée en cours d’UML.
      \item Le NAO se remet automatiquement à son heure d’usine lorsqu’il n’a pas été rechargé depuis trop longtemps (on soupçonne les batteries d’être usées).
      \item Certaines notions et conventions du langage python étant inconnues à Thibault, de nombreuses fonctions ont été développé inutilement (accesseurs sur les classes).
      Florian a donc pris l’initiative de briefer l’équipe sur ces points.\\
    \end{itemize}



  \subsection{Semaine du 30 mars au 4 mai}
  \label{sub:Semaine du 30 mars au 4 mai}
    \begin{itemize}
      \item Trouver des méthodes efficaces de détection des éléments du jeu de pong (il arrive encore que certains éléments ne soient pas détectés parfois)
      \item Le NAO a tendance à chauffer rapidement quand il est en position de jeu, ce qui nous oblige à devoir attendre pour reprendre les tests sur ce dernier.
      \item Trouver comment aborder le sujet du projet pour la présentation devant les lycéens et essayer d’adapter son discours en conséquence n’a pas été si aisé.
      \item Nous avons fait le choix d’utiliser le framework Reveal.js pour la présentation de notre PAO devant les lycéens.
      Cependant comme nous ne l’avions pas utilisé depuis un moment, cela a nécessité un certain temps de réadaptation.
      \item De nombreux problèmes, maintenant résolus, se sont présentés à nous pendant le développement de la Reconnaissance V2:
            \begin{itemize}
              \item afin de de détecter les éléments du jeu, la technique utilisée consiste à faire une moyenne du niveau de gris sur chaque ligne ou colonne.
              Si le résultat obtenu contient des valeurs comprises entre deux seuils paramétrés et pendant distance cohérente, nous pouvons en déduire la présence d’un élément du jeu.
              Un des premiers problèmes auquel nous avons été confronté a été de déterminer ces seuils.
              \item nous avons tout d’abord utilisé les images fournies par M.Delestre afin de détecter les éléments du pong.
              Il est cependant rapidement apparu que la résolution ainsi que la trop forte saturation en gris des l’images faussaient la plupart de nos résultats.
              \item après quelques tests sur le NAO, nous nous sommes rendu compte que l’apparence du pong avait été modifié depuis la capture des images que nous possédions.
              Cela impactait donc de manière assez conséquente le travail accompli et nous avons du refaire quelques prises.
              \item afin de régler les problèmes de seuil, une solution efficace consistait à passez à 0 les niveaux de gris inférieurs à 128, et monter à 255 les autres.
              Cependant le temps de calcul nécessaire à cette méthode rendait impossible l’application du module de reconnaissance sur le NAO.
            \end{itemize}
      \item Le main a été développé avant que le module de reconnaissance ne soit terminé.
      Il n’a donc pas pu être testé lors de sa conception.
      \item Le placement du NAO devant l’écran étant souvent fastidieux (nous nous savions jamais réellement si le robot était en mesure de détecter le terrain),
      un module affichant la vision du robot en temps réel a été conçu.\\
    \end{itemize}
\pagebreak
