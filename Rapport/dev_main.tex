\subsection{Main}
\label{sub:Main}

  \par Le script main permet l'execution de l'ensemble du projet.
  Il prend en paramètre une adresse IP sous la forme d'une chaine de caractères, et eventuellement un port.
  Dans le cas ou le port ne serait pas renseigné, un port par défaut est attribué.
  Cette adresse IP correspond à celle du NAO sur le réseau local.\\
  La première étape consiste donc à initialiser le robot, et lui appliquer sa position de jeu.\\
  Ensuite, il faut déterminer la vitesse de déplacement en pixels par secondes de la barre controlée par le NAO.
  Pour cela, la capture de deux images, interrompues par une action du bras du NAO d'une durée suffisante (une seconde dans notre cas), puiss leur traitement est necessaire.

  \par Le script est composé de deux fonctions: getImage(nao, reconnaissance) et traitement(nao, reconnaissance).\\
  La fonction getImage(nao, reconnaissance) permet, à partir d'une classe nao et reconnaissance initialisée.
