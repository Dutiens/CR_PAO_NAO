\subsubsection{Détection terrain}
\label{subs:Détection terrain}
\par L'utilisation qui avait été faite d'OpenCV pour la détection du contour du
terrain par la précédente équipe n'était que
peu précise, peu rapide et renvoyait les contours extérieurs du terrain. Nous avons donc
travaillé sur la récupération des contours intérieurs du terrain de manière
suffisament sûre et rapide en accord avec les contraintes de temps d'exécution du projet.
\par Nous avons donc réussi à faire en sorte d'obtenir ce que nous voulions en
un temps suffisament faible pour permettre au robot d'agir dans un temps
acceptable. Notre méthode consiste en la détection à l'aide d'OpenCV d'un
rectangle ayant une taille minimale fixée afin de ne pas détecter les éléments
du terrain, on découpe grossièrement l'image de sorte à inclure le rectangle
détecté, puis on répète l'opération jusqu'à ne plus détecter de rectangle. Enfin
on recadre le dernier rectangle obtenu de sorte à ne plus avoir de
perspective sur l'image et faciliter le traitement.

